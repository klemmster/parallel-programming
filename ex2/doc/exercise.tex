%%%%%%%%%%%%%%%%%%%%%%%%%%%%%%%%%%%%%%%%%%%%%%%%%%%%%%%%%%%%%
%% HEADER
%%%%%%%%%%%%%%%%%%%%%%%%%%%%%%%%%%%%%%%%%%%%%%%%%%%%%%%%%%%%%
\documentclass[a4paper,twoside,11pt]{article}

%% Language %%%%%%%%%%%%%%%%%%%%%%%%%%%%%%%%%%%%%%%%%%%%%%%%%
\usepackage[T1]{fontenc}

%\usepackage[ansinew]{inputenc}
\usepackage[utf8]{inputenc}	%supports Umlaute
\usepackage{german, ngerman}
\usepackage{color}

\usepackage{lmodern} %Type1-font for non-english texts and characters

%% Packages for Graphics & Figures %%%%%%%%%%%%%%%%%%%%%%%%%%
\usepackage{graphicx} %%For loading graphic files
\usepackage{multicol}

%% Math Packages %%%%%%%%%%%%%%%%%%%%%%%%%%%%%%%%%%%%%%%%%%%%
\usepackage{amsmath}
\usepackage{amsthm}
\usepackage{amsfonts}
\usepackage{amssymb}

%% Line Spacing %%%%%%%%%%%%%%%%%%%%%%%%%%%%%%%%%%%%%%%%%%%%%
\usepackage[parfill]{parskip}    % Activate to begin paragraphs with an empty line rather than an indent

%% Other Packages %%%%%%%%%%%%%%%%%%%%%%%%%%%%%%%%%%%%%%%%%%%
\usepackage{a4wide} %%Smaller margins = more text per page.
\usepackage{fancyhdr} %%Fancy headings

%%%%%%%%%%%%%%%%%%%%%%%%%%%%%%%%%%%%%%%%%%%%%%%%%%%%%%%%%%%%%
%% DOCUMENT
%%%%%%%%%%%%%%%%%%%%%%%%%%%%%%%%%%%%%%%%%%%%%%%%%%%%%%%%%%%%%
\begin{document}

\pagestyle{fancyplain}

%% Title Page %%%%%%%%%%%%%%%%%%%%%%%%%%%%%%%%%%%%%%%%%%%%%%%
\title{Parallel Programming Assignment \#2} 
\author{Marcel Karsten -- 343619,\\ Patrick Lorenz -- 341922,\\ Richard Klemm -- 343635 }
\date{Due: Thursday, 24th May 2012} %%If commented, the current date is used.
\maketitle

%% Header on top of every page (yes, also on the title page!)
\rhead{Parallel Programming - TU Berlin, SS/12}
\lhead{}
\renewcommand{\headrulewidth}{0px}


%%%%%%%%%%%%%%%%%%%%%%%%%%%%%%%%%%%%%%%%%%%%%%%%%%%%%%%%%%%%%

\section{Exercise 1 - Sequential Overspecification}
Input of the algorithm are two strings (a, b) and a minimum matching length (K). The algorithm compares sequence \textit{a} to sequence \textit{b} and provides all matching substrings with at least K characters.

The first part of the algorithm (first nested for loops) fills the cells of the matrix L. Important are the diagonals that represent a specific starting points in the sequences. Every cell (i, j) of the matrix now indicates if the characters a[i] and b[j] match and how long the matching substring towards that point is. The diagonal on the bottom left (see example matrix) with values from 1 to 4 shows the matching substring "`Welt"` that is part of both sequences. The same stands for the diagonal on the top right (see example matrix) going from 1 to 5 with the substring "`Hallo"'.

The second part now selects the matching substrings that have at least K characters and prints the indices of the sequences and the matching length.


The example inputs a="'HalloWelt"' and b="'WeltHallo"' yield the following matrix 

$$L=
\begin{pmatrix}
	0 & 0 & 0 & 0 & 1 & 0 & 0 & 0 & 0 \\
	0 & 0 & 0 & 0 & 0 & 2 & 0 & 0 & 0 \\
	0 & 0 & 1 & 0 & 0 & 0 & 3 & 1 & 0 \\
	0 & 0 & 1 & 0 & 0 & 0 & 1 & 4 & 0 \\
	0 & 0 & 0 & 0 & 0 & 0 & 0 & 0 & 5 \\
	1 & 0 & 0 & 0 & 0 & 0 & 0 & 0 & 0 \\
	0 & 2 & 0 & 0 & 0 & 0 & 0 & 0 & 0 \\
	0 & 0 & 3 & 0 & 0 & 0 & 1 & 1 & 0 \\
	0 & 0 & 0 & 4 & 0 & 0 & 0 & 0 & 0
\end{pmatrix}
$$

\section{Exercise 2 - Parallel Algorithm Design}
\textbf{(a) Foster's design methodology}

Foster's Methodology:
\begin{enumerate}
	\item Partitioning
	\item Communication
	\item Agglomeration
	\item Mapping
\end{enumerate}

\textbf{Partitioning}

domain decomposition, functional decomposition; good: minimize redundant computations and redundant data storage, primitive tasks are the same size, number of tasks increase with problem size

Memory consumption: higher k $\rightarrow$ lower memory consumption

\textbf{Communication}

local communication, global communication; good: comm. is balanced among tasks, tasks can communicate concurrently, tasks can perform computations concurrently

\textbf{Agglomeration}

-

\textbf{Mapping}

-

\textbf{(b) OpenMP}

See source of classes OpenMPAlgorithm and OpenMPAlgorithmTasks.

\textbf{(c) Optimization}

See source of classes SequentialOptimizedAlgorithm and SequentialOptimizedAlgorithm2.

\section{Exercise 3 - Benchmarking}

Results here...

\end{document}
